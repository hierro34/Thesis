\chapter{Conclusions and Future Work}

\section{Conclusions}

The DMP is are desirable feature to be enjoyed by the numerical methods and it has been seen that it can be achieved by using different techniques: continuous Galerkin, discontinuous Galerkin $hp$-

\section{Conclusions 1}
In this chapter we introduce the linear stabilization technique for continuous FE
discretizations of time-dependent transport problems which belongs to the family of local 
projection stabilization techniques. In particular, we consider a Scott-Zhang-type projector which is 
well-defined for $L^1(\Omega)$ functions, extending the ideas in \cite{badia_stabilized_2012} to convection stabilization. 
Stability and numerical analyses for the linear transport problem are carried out. 

Further, we design a weighting of the aforementioned linear stabilization such that, 
when combined with a FE discretization with a DMP (usually
attained via a shock-capturing technique), it does not spoil the monotonicity properties. 
It is attained by switching off the linear stabilization around shocks.

Next, we have proposed different nonlinear stabilization (shock-capturing) schemes based on 
artificial viscosity, in order to reduce or even eliminate local oscillations around shocks/discontinuities. 
In particular, we have used a definition of the artificial viscosity based on boundary gradient jumps (bGJV),
following the original work of Burman in \cite{burman_nonlinear_2007}, and another one based on
nodal jumps (nGJV). For the nodal method, we have proved a salient strong DMP property for multidimensional 
time-dependent transport problems.

Finally, a complete set of numerical experiments is included. On one hand, we check experimentally
the theoretical monotonicity properties of the weighting formulations and the nonlinear stabilization. Next, gradient-jump shock-capturing methods (with different linear stabilizations) are compared against residual-based and entropy-based formulations, in order to show its performance. The results obtained with the nGJV scheme are remarkably good, with oscillation-free solutions in different tests.

Future work will be the extension of GJV methods to high order and/or discontinuous Galerkin formulations.
Further, since these methods do not rely on entropy functions, they can
also be extended to CDR problems, in order to properly capture boundary and internal layers. 

\section{Future Work}
In the latter years the amount of methods that can be proved to enjoy the DMP have proliferated. The contributions to this field have come, a part from this thesis, from the work of several different authors such as  Badia, Barrenechea, Bonilla, Burman, Guermond, Karakatsani, Kuzmin,  Nazarov, Popov, Shadid, Yang  \cite{barrenechea_blending_2015,guermond_second-order_2014,guermod_nazarov_2014,kuzmin_gradient-based_2016,jesus} among others. This is good news for implementing methods that are robust in terms of positive preserving. Even though the different techniques have been implemented for different methods linear continuous Galerkin, quadratic continuous Galerkin, linear discontinuous Galerkin, Finite Volumes... Most of the analysis has been performed on Convection-diffusion(-reaction) and Burgers' equations since they are some of the simpliest equations and they allow a monotonicity analysis of artificial diffusion techniques under certain mesh assumptions are possible. So it still last a hard work of generalizing such properties for more complex equations such as ??????. In order to do so, one of the most powerful techniques are the ones introduced by Codina in \cite{codina_discontinuity-capturing_1993} and improved by Kuzmin et al. in \cite{kuzmin_flux-corrected_2005} and Badia and Bonilla \cite{jesus} for continuous Galerkin and the contribution of Chapter \ref{chap-paper4} for discontinuous Galerkin. This strategy relies on the properties of the matrix instead of depending of the equations what make them easy to extend for intricate sets of equations. A first attempt to do so have been performed by Kuzmin in \cite{???} with the XXXX equation and Badia and Bonilla in \cite{jesus}.\\

It has been shown in Chapter \ref{chap-paper4} that the Newton method is much more efficient than other nonlinear iterators in terms of number of iterations. It might be interesting to further investigate if there are other techniques that can also diminish the number of iteration without the need of constructing the jacobian. 