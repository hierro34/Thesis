\chapter{Conclusions and Future Work}

\section{Conclusions}

The DMP is are desirable feature to be enjoyed by the numerical methods and it has been seen that it can be achieved by using different techniques: continuous Galerkin, discontinuous Galerkin $hp$-

\section{Future Work}
In the latter years the amount of methods that can be proved to enjoy the DMP have proliferated. The contributions to this field have come, a part from this thesis, from the work of several different authors such as  Badia, Barrenechea, Bonilla, Burman, Guermond, Karakatsani, Kuzmin,  Nazarov, Popov, Shadid, Yang  \cite{barrenechea_blending_2015,guermond_second-order_2014,guermod_nazarov_2014,kuzmin_gradient-based_2016,jesus} among others. This is good news for implementing methods that are robust in terms of positive preserving. Even though the different techniques have been implemented for different methods linear continuous Galerkin, quadratic continuous Galerkin, linear discontinuous Galerkin, Finite Volumes... Most of the analysis has been performed on Convection-diffusion(-reaction) and Burgers' equations since they are some of the simpliest equations and they allow a monotonicity analysis of artificial diffusion techniques under certain mesh assumptions are possible. So it still last a hard work of generalizing such properties for more complex equations such as ??????. In order to do so, one of the most powerful techniques are the ones introduced by Codina in \cite{codina_discontinuity-capturing_1993} and improved by Kuzmin et al. in \cite{kuzmin_flux-corrected_2005} and Badia and Bonilla \cite{jesus} for continuous Galerkin and the contribution of Chapter \ref{chap-paper4} for discontinuous Galerkin. This strategy relies on the properties of the matrix instead of depending of the equations what make them easy to extend for intricate sets of equations. A first attempt to do so have been performed by Kuzmin in \cite{???} with the XXXX equation and Badia and Bonilla in \cite{jesus}.\\

It has been shown in Chapter \ref{chap-paper4} that the Newton method is much more efficient than other nonlinear iterators in terms of number of iterations. It might be interesting to further investigate if there are other techniques that can also diminish the number of iteration without the need of constructing the jacobian. 