\chapter{Monotonicity preserving methods for discontinuous Galerkin Methods}

We define a \textit{nonnegative type} matrix as any $N\times N$ matrix $A=\{a_{ij}\}$ such that
\begin{align}
&a_{ij} \leq 0  & & {\rm for}\, i\neq j, \, i=1,\cdots,N_R, j=1,\cdots,N \\
&\sum_{j=1}^{N} a_{ij} \geq 0 & &{\rm for}\, i=1,\cdots,N_R
\end{align}
We define $A_R = \{a_{ij}\}$ with $i,j=1,\cdots, N_R$.
In \cite{codina_discontinuity-capturing_1993}, Codina proved that
\begin{theorem}[Sufficient condition for DMP]
	If A is of \textit{non-negative type}, $A_R$ is non singular and $b_i\leq 0$ for $i=1,\cdots,N_r$ then 
	\begin{align*}
	Ax=b & \Rightarrow & \exists m\in \{N_R+1,\cdots,N\} \, s.t. \, \max_{i\in\{1,\cdots,N\}}\{x_i\}=x_m
	\end{align*}
\end{theorem}

Somehow this is changed to the column property and dG methods already have such property.

\section{Introduction}

The discontinuous Galerkin scheme (art 009).\\

The DMP property. Artificial diffusion techniques (art 009).\\

The DMP property. Properties of the matrices (kuzmin 2015-2016).\\

The unexplored field of dG methods\\

The aim of this paper.\\

The structure of the paper.\\

\section{The Convection-Diffusion Problem and its Discretization}

The problem we are going to solve.\\

The potential apparition of wiggles around layers.\\

\subsection{Notation}

$\ndofs$\\
$\Kp$\\
$\setneighbours$\\
$\shapeik$\\
$\support{i}$\\



\subsection{Weak form and the interior penalty discontinuous Galerkin approximation}
The symmetric interior penalty discontinuous Galerkin (IPDG) method with nonlinear artificial viscosity stabilization reads as
\begin{align}\label{eq-dscrtpbm}
{\rm Find}\quad u_h\in V_h \quad {\rm such}\,{\rm that }\quad a_h(u_h,v_h) = l(v_h) \quad \forall v_h\in V_h,
\end{align}
where 
\begin{align}\label{eq-bform}
\begin{split}
a_h&(u_h,v_h) =  \displaystyle\sum\limits_{\elem \in \thm} \int_\elem\left( \mu   \nabla u_h \nabla v_h - 
u_h  \beta \cdot \nabla v_h\right) \\
& - \sum\limits_{F\in \setfaces}  \int_{F} \mu (\jump{u_h}\mean{ \nabla v_h} + \mean{\nabla u_h} \jump{v_h})    +  \sum\limits_{F\in \setfaces}  \int_{F} c^{\rm ip} {h}_F^{-1} \mu\jump{  u_h}\jump{  v_h}   \\
&+ \sum\limits_{F\in \setfaces{ \setminus\partial\Omega^-}}  \int_{F}   \mean{\beta u_h}\jump{v_h} + c^{{\rm bms}} \sum\limits_{F\in \setintfaces}  \int_{F}  |\beta|  \jump{u_h} \jump{v_h} %\nonumber
\end{split}
\end{align}
and 
\begin{align}\label{eq-rhs}
\begin{split}
l(v_h) = & \displaystyle\sum\limits_{\elem \in \thm} (f_h,v_h)_K - \sum\limits_{F\in\partial \Omega^-}\int_{F} \beta \cdot n_{\partial\Omega}g_h v_h  \\
& +\sum\limits_{F\in \partial \Omega}\int_F \mu g_h \mean{ \nabla v_h} \cdot n_{\partial\Omega}
-\sum\limits_{F\in \partial \Omega}\int_F c^{\rm ip} h_F^{-1} \meanhar{\mu+\varepsilon_h(u_h)} g_h v_h , 
\end{split}
\end{align}

$\spacebilinearform$\\
$\rhs$

\section{The Discrete Maximum principle}

In this section we will introduce the desired properties that we will expect our discrete problem to fulfill.

\subsection{The local discrete extrema}

Recovering the notation in \cite{badia_discrete_2015}, we will define the local discrete extrema as follows.

 \begin{definition}[local discrete extremum]
 	The function $u_h\in V_h$ has a local discrete minimum (resp., maximum)  on node $x_i$ in $K$ if $u_i^K \leq u_h(x)$ (resp., $u_i^K \geq u_h(x)$) $\forall x\in \Omega_i$.
 \end{definition}
 
 In particular, when using linear elements, it is this is equivalent to say that $u_i^K\leq u_j^\Kp$ for any pair $\{j,\Kp\}\in \setneighbours(i,K) $.

\subsection{The space matrix integration}

Given a matrix-vector problem of the form:

\begin{align*}
 \mathbb{Au} = \mathbb{f}
\end{align*}

with $ \mathbb{A}\in \mathbb{R}^\ndofs\times\mathbb{R}^\ndofs$ and $ \mathbb{u}, \mathbb{f}\in\mathbb{R}^\ndofs$ corresponding to the space problem, that is, given $a=\iktoa(i,K)$ and $b=\iktoa(j,\Kp)$, the components of $ \mathbb{A}$ and $f$ are defined by $A_{ab}=\spacebilinearform(\shapeik,\shape{j}{\Kp})$ and $f_a=\rhs(\shapeik)$. 

\begin{definition}
	We will say that the matrix $ \mathbb{A}$ enjoys the \textit{Discrete Maximum Principle} (DMP) if, given for any interior node $x_i$ and any $K\in\support{i}$, the following hold true: if $u_h$ has a local maximum (resp. minimum) on $x_i$ in $K$ and $a=\iktoa(i,K)$, then $f_a \geq 0$ ($f_a\leq0$ resp.).
\end{definition}

It is clear that, since we can control the sign of $ \mathbb{f}$ by knowing the sign of the source term $f$, it is clear that this is to say that given a negative (positive) source term the solution of the discrete problem will not present any local discrete maximum (minimum) in the interior of the domain.


As it is shown in \cite{Jesus}, in order to ensure a DMP it is enough to construct a matri



