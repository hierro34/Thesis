\chapter{Introduction and Motivation}
\label{chap-intro}


How to deal with the shocks. In the {author's} masther thesis \ref{masterthesis}, it can be appreciated how some very different shock capturing techniques perform with diferent methods, continuous and discontinuous Galerkin with low and high order. The test were performed for one-dimensional cases but they already shed light some how effective some of these techniques were in different paradigms.

One important revelation was that, contrary to what was expected, the discontinuous Galerkin techniques were not more effective in capturing the discontinuities than the continuous method. This was due to the fact that the interior penalty terms would not let the method to capture a pure discontinuity between two elements and this induce spurious oscillations around the discontinuity or sharp layer.

In \ref{masterthesis} the Gradient Jump Viscosity (GJV) method was already introduced for the one-dimensional case as an adaptation of the method proposed by \cite{burman_nonlinear_2007} for the Burgers' equation.

\section{Structure of the document}
Chapters \ref{chap-paper1}, \ref{chap-paper2} and \ref{chap-paper3} are based on the \comment{author's} articles \cite{badia_stabilized_2012}, \cite{badia_discrete_2015} and \cite{paper3} respectively. The three chapters follow the structure of the publications and, thus, are selfcontained. Some of the notation used might slightly differ from one chapter to another according to the necessities of each analysis. In this sense, on \ref{chap-paper3}, the elements of the mesh are ordered while in the rest they are not, and there is some abuse of notation on the solution space: it is called $\thn$ trough the document but it stnds for different concretizations in each of them (continuous or discontinuous). 

\section{Things}
On the continuous Galerkin paradigm, the need of an extra linear stabilization seems to be obvious and the challenge faced in \ref{chap-paper1} will be how to blend the linear and nonlinear stabilization in such a way that one preserves the DMP property. The discontinuous Galerkin, on the other hand, do already have implicit linear stabilization in its definition so the artificial viscosity can be directly applied on the top of it.

The origin of continuous and discontinuous Galerkin. The amount of work related to the development of methods that enjoy the DMP for cG in comparison with dG. In this sense we feel that the present work is a remarkable contribution to the field since it includes methods that go from the analysis of the possible artificial diffusion techniques that would enjoy such properties to the implementation of graphviscosity methods that allow the method to stricly fulfil the property for any kind of underlying mesh.

In all of the cases a gradient jump viscosity shock detectors are implemented. The definition of those detectors can be generalized and applied either for continuous and discontinuous methods. This property is very interesting specially for hybrid meshes such as the continuous-discontinuous Galerkin method introduced in \cite{badia_adaptive_2013} for dealing with hanging nodes in adaptive meshes. In this sense, in chapter \ref{chap-paper3}, there is a possible approach to deal with the adaptive paradigm.

The present thesis aims to be the first compilation of different applications that take use of the Gradient Jump Viscosity shock detectors in order to implement Discrete Maximum Principle enjoying methods.